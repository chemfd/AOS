% Options for packages loaded elsewhere
\PassOptionsToPackage{unicode}{hyperref}
\PassOptionsToPackage{hyphens}{url}
%
\documentclass[
]{book}
\usepackage{lmodern}
\usepackage{amssymb,amsmath}
\usepackage{ifxetex,ifluatex}
\ifnum 0\ifxetex 1\fi\ifluatex 1\fi=0 % if pdftex
  \usepackage[T1]{fontenc}
  \usepackage[utf8]{inputenc}
  \usepackage{textcomp} % provide euro and other symbols
\else % if luatex or xetex
  \usepackage{unicode-math}
  \defaultfontfeatures{Scale=MatchLowercase}
  \defaultfontfeatures[\rmfamily]{Ligatures=TeX,Scale=1}
\fi
% Use upquote if available, for straight quotes in verbatim environments
\IfFileExists{upquote.sty}{\usepackage{upquote}}{}
\IfFileExists{microtype.sty}{% use microtype if available
  \usepackage[]{microtype}
  \UseMicrotypeSet[protrusion]{basicmath} % disable protrusion for tt fonts
}{}
\makeatletter
\@ifundefined{KOMAClassName}{% if non-KOMA class
  \IfFileExists{parskip.sty}{%
    \usepackage{parskip}
  }{% else
    \setlength{\parindent}{0pt}
    \setlength{\parskip}{6pt plus 2pt minus 1pt}}
}{% if KOMA class
  \KOMAoptions{parskip=half}}
\makeatother
\usepackage{xcolor}
\IfFileExists{xurl.sty}{\usepackage{xurl}}{} % add URL line breaks if available
\IfFileExists{bookmark.sty}{\usepackage{bookmark}}{\usepackage{hyperref}}
\hypersetup{
  pdftitle={CH3/40227 Advanced Spectroscopic Techniques},
  pdfauthor={Fiona Dickinson},
  hidelinks,
  pdfcreator={LaTeX via pandoc}}
\urlstyle{same} % disable monospaced font for URLs
\usepackage{longtable,booktabs}
% Correct order of tables after \paragraph or \subparagraph
\usepackage{etoolbox}
\makeatletter
\patchcmd\longtable{\par}{\if@noskipsec\mbox{}\fi\par}{}{}
\makeatother
% Allow footnotes in longtable head/foot
\IfFileExists{footnotehyper.sty}{\usepackage{footnotehyper}}{\usepackage{footnote}}
\makesavenoteenv{longtable}
\usepackage{graphicx,grffile}
\makeatletter
\def\maxwidth{\ifdim\Gin@nat@width>\linewidth\linewidth\else\Gin@nat@width\fi}
\def\maxheight{\ifdim\Gin@nat@height>\textheight\textheight\else\Gin@nat@height\fi}
\makeatother
% Scale images if necessary, so that they will not overflow the page
% margins by default, and it is still possible to overwrite the defaults
% using explicit options in \includegraphics[width, height, ...]{}
\setkeys{Gin}{width=\maxwidth,height=\maxheight,keepaspectratio}
% Set default figure placement to htbp
\makeatletter
\def\fps@figure{htbp}
\makeatother
\setlength{\emergencystretch}{3em} % prevent overfull lines
\providecommand{\tightlist}{%
  \setlength{\itemsep}{0pt}\setlength{\parskip}{0pt}}
\setcounter{secnumdepth}{5}
\usepackage{booktabs}
\usepackage{amsthm}
\makeatletter
\def\thm@space@setup{%
  \thm@preskip=8pt plus 2pt minus 4pt
  \thm@postskip=\thm@preskip
}
\makeatother
\usepackage[]{natbib}
\bibliographystyle{apalike}

\title{CH3/40227 Advanced Spectroscopic Techniques}
\author{Fiona Dickinson}
\date{2021-02-01}

\begin{document}
\maketitle

{
\setcounter{tocdepth}{1}
\tableofcontents
}
\hypertarget{ch340227-advanced-spectroscopic-techniques}{%
\chapter*{CH3/40227 Advanced Spectroscopic Techniques}\label{ch340227-advanced-spectroscopic-techniques}}
\addcontentsline{toc}{chapter}{CH3/40227 Advanced Spectroscopic Techniques}

\hypertarget{welcome-preliminary-infomation}{%
\section*{Welcome \& Preliminary Infomation}\label{welcome-preliminary-infomation}}
\addcontentsline{toc}{section}{Welcome \& Preliminary Infomation}

Welcome to the coursepage for CH30227 \& CH40227 Advanced Spectroscopic Techniques. The notes have been prepared in a package called BookDown for RStudio so that the equations are accessible to screen readers. However, by providing the notes as a .html webpage I can also embed short videos to further describe some of the topics. Further you can download the material in a format that suits you (either pdf or epub) to view offline, or change the way this document appears for ease of reading.

The course is entirely taught by Dr Fiona Dickinson.

\hypertarget{prerequisite-knowledge}{%
\subsection*{Prerequisite knowledge}\label{prerequisite-knowledge}}
\addcontentsline{toc}{subsection}{Prerequisite knowledge}

The course relies extensively on concepts from CH30129 (or CH30217) Photochemistry as well as material from the spectroscopy section of CH10137/8, and the quantum mechanics of CH20151/2. However more generic skills, such as experiment planning, and drawing conclusions on data will form an important strand to this course.

\hypertarget{assessment-for-this-course}{%
\subsection*{Assessment for this course}\label{assessment-for-this-course}}
\addcontentsline{toc}{subsection}{Assessment for this course}

The course is assessed by a `1 hour' exam, this is to say it should take an hour to complete, exam details will be shared centrally later in the semester.

The exam format will contain 2 x 25 mark questions. The questions will likely contain two or more of the following: data to examine, description of a spectrometer or spectrometer component, and design of experiments.

\hypertarget{feedback-for-this-course}{%
\subsection*{Feedback for this course}\label{feedback-for-this-course}}
\addcontentsline{toc}{subsection}{Feedback for this course}

Much of the work will be peer based in small groups, and you will provide feedback to each other. In wrapping up these discussions I will also provide feedback to discussions.

I appreciate your understanding that feedback may be delayed during these uncertain times as I have childcare responsibilities.

I will not provide model answers for past exam questions, this is because there will be multiple ways in which marks may be achieved. Instead I will happily provide feedback on your attempts at past papers. All I ask is:

\begin{itemize}
\tightlist
\item
  papers are received in good time
\item
  when you attempt papers you try and replicate the exam conditions (i.e.~do it alone, in one sitting in a limited period)
\item
  you do not submit more than one past paper at a time (I am happy to go through more than one feedback cycle, but want you to reflect on the feedback you have received)
\item
  you highlight sections where you particularly want feedback
\item
  you provide the file as a .pdf, with the file name containing your username, the year of paper attempted and the unit code
\item
  please space out work enough so that I can write feedback
\end{itemize}

\hypertarget{week-1-tasks}{%
\section*{Week 1 tasks}\label{week-1-tasks}}
\addcontentsline{toc}{section}{Week 1 tasks}

I set a number of week 1 tasks.

\begin{itemize}
\item
  For the `spectrometer components document' on Teams, please divide the work between you, and write a few sentences (and include figures if relevant) for your chosen component(s). This is a shared document and is hosted on Teams because I can't work out how to do it on Moodle\ldots{} (the CH40227 class will act as editors to this body or work).
\item
  Think about the spectroscopic techniques you have used previously (you should all have used at least one of UV/vis, IR and fluorescence), think about what components may have been in the `beige box' and how these differ between different spectrometers\ldots{}
\item
  Remember to let me know if you want particular groups in the zoom chat C(CH30227 only)
\end{itemize}

\hypertarget{report-errors}{%
\section*{Report errors}\label{report-errors}}
\addcontentsline{toc}{section}{Report errors}

If you spot any errors, please message me in Teams or (if this works), report on the error log below

Loading\ldots{}

\hypertarget{edit-log}{%
\section*{Edit log}\label{edit-log}}
\addcontentsline{toc}{section}{Edit log}

Initial commit 010221

  \bibliography{book.bib,packages.bib}

\end{document}
